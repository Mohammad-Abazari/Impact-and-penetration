%  !TEX TS-program = pdflatex
% !TEX encoding = UTF-8 Unicode

% This is a simple template for a LaTeX document using the "article" class.
% See "book", "report", "letter" for other types of document.

\documentclass[]{article} % use larger type; default would be 10pt

\usepackage[utf8]{inputenc} % set input encoding (not needed with XeLaTeX)

%%% Examples of Article customizations
% These packages are optional, depending whether you want the features they provide.
% See the LaTeX Companion or other references for full information.

%%% PAGE DIMENSIONS
%\usepackage{geometry} % to change the page dimensions
%\usepackage[top=0.9in,bottom=0.9in, left=.75in,right=0.75in, includefoot]{geometry}
%\geometry{a4paper} % or letterpaper (US) or a5paper or....
% \geometry{margin=0.75in} % for example, change the margins to 2 inches all round
% \geometry{landscape} % set up the page for landscape
%   read geometry.pdf for detailed page layout information

\usepackage{graphicx} % support the \includegraphics command and options
\usepackage{mathtools,tikz}

% \usepackage[parfill]{parskip} % Activate to begin paragraphs with an empty line rather than an indent
%\usepackage{fourier}
%\usepackage{charter}
%\usepackage{bookman}
%\usepackage{newcent}

%\usepackage[sc]{mathpazo} % Palatino with smallcaps
%\usepackage[scaled]{helvet}
%% Helvetica, scaled 95%
%\usepackage{eulervm}
%% Euler math

\usepackage{palatino}


%%% PACKAGES
\usepackage{booktabs} % for much better looking tables
\usepackage{array} % for better arrays (eg matrices) in maths
\usepackage{paralist} % very flexible & customisable lists (eg. enumerate/itemize, etc.)
\usepackage{verbatim} % adds environment for commenting out blocks of text & for better verbatim
\usepackage{subfig} % make it possible to include more than one captioned figure/table in a single float
% These packages are all incorporated in the memoir class to one degree or another...
\usepackage[pagebackref=true, colorlinks, linkcolor=blue, citecolor=blue, urlcolor=blue] {hyperref}%
%%% HEADERS & FOOTERS
\usepackage{fancyhdr,cleveref} % This should be set AFTER setting up the page geometry
\pagestyle{fancy} % options: empty , plain , fancy
\renewcommand{\headrulewidth}{0pt} % customise the layout...
\lhead{}\chead{}\rhead{}
\lfoot{}\cfoot{\thepage}\rfoot{}

%%% SECTION TITLE APPEARANCE
\usepackage{sectsty}
\allsectionsfont{\sffamily\mdseries\upshape} % (See the fntguide.pdf for font help)
% (This matches ConTeXt defaults)
\usepackage[round]{natbib}

%%% ToC (table of contents) APPEARANCE
\usepackage[nottoc,notlof,notlot]{tocbibind} % Put the bibliography in the ToC
\usepackage[titles,subfigure]{tocloft} % Alter the style of the Table of Contents
\renewcommand{\cftsecfont}{\rmfamily\mdseries\upshape}
\renewcommand{\cftsecpagefont}{\rmfamily\mdseries\upshape} % No bold!

\title{Final Exam: Impact and Penetration}
\author{Mohammad Abazari}
\date{} % Activate to display a given date or no date (if empty),
         % otherwise the current date is printed 

\begin{document}
\maketitle

\section*{Problem 1}
A protective wall is struck by a rigid ogive projectile traveling 1,000 m/s. Provide a safe and economic wall design. Material and projectile properties provided in \ref{tab02}.

% Table generated by Excel2LaTeX from sheet 'Sheet1'
\begin{table}[htbp]
  \centering
  \caption{Material and projectile properties.}
  \resizebox{\textwidth}{!}{%
    \begin{tabular}{cccc}
    \multicolumn{4}{c}{Material properties.} \\
\midrule    Material Type & {$\rho_0$} & {Yield strength} & {Unit cost} \\
  & {Ton/m$^3$} & {$\sigma_y$ or $f_c'$, MPa} & {\$/Ton} \\
\midrule    Steel(A572 Grade 60)\citep{a572} & 7.850  & {400} & {1,800} \\
    Concrete(UHPC$^*$)\citep{Akhnoukh2021} & 2.500  & {90} & {1,200}   \\
        \bottomrule
        $^*$ Ultra high performance concrete.
    \end{tabular}%
}
    \begin{tabular}{cccccc}
\multicolumn{6}{c}{Projectile properties.} \\
    \midrule
    Material Type &{mass, kg} & {Length, m} &{$s$, m} &{$d$, m} & {{$\mu_\mathrm{m}$}} \\
    \midrule
    Steel & 0.2   & 0.05  & 0.1   & 0.04  & 0.2 \\
        \bottomrule
    \end{tabular}%
  \label{tab02}%
\end{table}%

\section*{Solution}
The objective here is to prevent perforation in an economic way. There are multiple solutions, due to the variety of methods available.
\subsection*{Concrete panel}
\cite{forrestal1994empirical,forrestal1996penetration,frew2006effect,frew1998penetration} proposed the following formula for projectile depth of penetration\footnote{DOP.},
\begin{equation}\label{eq01}
  \begin{aligned}
  h_{\text {pen }}=&\frac{\left(L+0.5 k^{\prime} d\right)}{2 N}\left(\frac{\rho_p}{\rho_0}\right) \ln \left[1+\frac{N \rho_0 V_1^2}{S f_{\mathrm{c}}}\right]+2 d \\
  =&\frac{2 M}{\pi d^2 \rho_0 N} \ln \left[1+\frac{N \rho_0 V_1^2}{S f_{\mathrm{c}}}\right]+2 d
  \end{aligned}
\end{equation}
\begin{equation}\label{eq02}
k^{\prime}=\left(4 \psi^2-\frac{4 \psi}{3}+\frac{1}{3}\right)(4 \psi-1)^{0.5}-4 \psi^2(2 \psi-1) \sin ^{-1}\left[\frac{(4 \psi-1)^{0.5}}{2 \psi}\right] 
  \end{equation}
  \begin{equation}\label{eq03}
  N=\frac{8 \psi-1}{24 \psi^2}; \quad V_1^2=\frac{V_0^2-\left({2 d}/({L+0.5 k^{\prime} d})\right)\left({S f_c}/{\rho_p}\right)}{1+N\left({2 d}/({L+0.0 k^{\prime} d})\right)\left({\rho_0}/{\rho_p}\right)} ; \quad S=82.6 \times\left(f_{\mathrm{c}} \times 10^{-6}\right)^{-0.544}
  \end{equation}

\cite{forrestal2003penetration} modified the above. The proposed formula are only applicable for
ogive-nosed projectiles while neglecting friction between projectile and target medium.
\begin{equation}\nonumber
  \begin{aligned}
  \psi & =\frac{0.1}{0.04} \rightarrow N=\frac{8(2.5)-1}{24(2.5)^2} \rightarrow=0.1267 \\
  S & =82.6 \times\left(90 \times 10^6 \times 10^{-6}\right)^{-0.544} \rightarrow=7.143 \\
  k^{\prime} & =\left(4(2.5)^2-\frac{4(2.5)}{3}+\frac{1}{3}\right)(4(2.5)-1)^{0.5}-4(2.5)^2(2(2.5)-1) \sin ^{-1}\left[\frac{(4(2.5)-1)^{0.5}}{2(2.5)}\right] \\
  & =1.650 \\
  V_1^2 & =\left[{(1000)^2-\left(\frac{2(0.04)}{0.05+0.5(1.650)(0.04)}\right)\left(\frac{(7.430)\left(90 \times 10^6\right)}{7850}\right)}\right]\\
 &\times \left[{1+(0.1267)\left(\frac{20.04)}{0.05+0.5(1.650)(0.04)}\right)\left(\frac{2500}{7850}\right)}\right]^{-1} \\
  & =886000 \rightarrow V_1=941.6 \mathrm{~m} / \mathrm{s} \\
  h_{\text {pen }} & =\frac{2(0.2)}{\pi(0.04)^2(2500)(0.1267)} \ln \left[1+\frac{(0.1267)(2500)(941.6)^2}{(7.143)\left(90 \times 10^6\right)}\right]+2(0.04) \\
  & =0.1711 \rightarrow \text { or } 17.11 \mathrm{~cm}
  \end{aligned}
  \end{equation}
\cite{chen2002deep,li2003dimensionless} extended the equations presented by \cite{forrestal1993penetration,forrestal1996penetration} to a dimensionless form applicable to arbitrary-nosed projectiles. According to \cite{chen2002deep,li2003dimensionless} maximum penetration depth in a thick plate is
\begin{equation}\label{eq04}
  \frac{h_{\mathrm{pen}}}{d}=\frac{2}{\pi} N \ln \left(1+\frac{I_0}{N}\right)
  \end{equation}
  \begin{equation}\label{eq05}
    I_0=\frac{M V_0^2}{N_1 S f_{\mathrm{c}} d^3} ; \quad N=\frac{M}{N_2 \rho_0 d^3} ; \quad S=72.0 \times\left(f_{\mathrm{c}} \times 10^{-6}\right)^{-0.5}
    \end{equation}
  
where $I_0$ and $N$ are defined as the impact and geometry functions, $N_1$ and $N_2$ account for projectile nose geometry as well as friction. For an arbitrary-nosed projectile, $N_1$ and $N_2$ are obtained through integrating the projectile nose profile function along the nose length. Explicit
expressions of $N_1$ and $N_2$ for ogive, conical, blunt, truncated ogive, hemispherical, and flat projectile noses are given \cite{chen2002deep}. For ogive nosed projectiles we have
\begin{equation}\nonumber
  \begin{aligned}
  & N_1=1+4 \mu_{\mathrm{m}} \psi^2\left[\left(\frac{\pi}{2}-\phi_0\right)-\frac{\sin 2 \phi_0}{2}\right] \\
  & N_2=N^*+\mu_{\mathrm{m}} \psi^2\left[\left(\frac{\pi}{2}-\phi_0\right)-\frac{1}{3}\left(2 \sin 2 \phi_0+\frac{\sin 4 \phi_0}{4}\right)\right] \\
  & N^*=\frac{1}{3 \psi}-\frac{1}{24 \psi^2}, \quad 0<N^* \leq \frac{1}{2} \\
  & \phi_0=\sin ^{-1}\left(1-\frac{1}{2 \psi}\right), \quad \text { where } \psi \geq \frac{1}{2}
  \end{aligned}
  \end{equation}

Inclusion of sliding friction coefficient makes this formula quite advantageous.
\begin{equation}\nonumber
  \begin{aligned}
  \phi_0 & =\sin ^{-1}\left(1-\frac{1}{2 \psi}\right) \rightarrow=0.9273 \text { or } 53.13^{\circ} \\
  N^* & =\frac{1}{3(2.5)}-\frac{1}{24(2.5)^2} \rightarrow=0.1267 \\
  N_1 & =1+4(0.2)(2.5)^2\left[\left(\frac{\pi}{2}-(0.9273)\right)-\frac{\sin 2(0.9273)}{2}\right] \rightarrow=1.817 \\
  N_2 & =(0.1267)+0.2(2.5)^2\left[\left(\frac{\pi}{2}-(0.9273)\right)-\frac{1}{3}\left(2 \sin 2(0.9273)+\frac{\sin 4(0.9273)}{4}\right)\right] \\
  & =0.1871\\
    S & =72.0 \times\left(\left(90 \times 10^6\right) \times 10^{-6}\right)^{-0.5} \rightarrow=7.589 \\
    I_0 & =\frac{(0.2)(1000)^2}{(1.8175)(7.589)\left(90 \times 10^6\right)(0.04)^3} \rightarrow=2.5172 \\
    N & =\frac{0.2}{(0.1871)(2500)(0.04)^3} \rightarrow=6.681 \\
    h_{\text {pen }} & =(0.04)\left(\frac{2}{\pi}(6.681) \ln \left(1+\frac{(2.5172)}{(6.681)}\right)\right) \\
    & =0.05440 \text { or } 5.440 \mathrm{~cm}
    \end{aligned}
    \end{equation}
while with no friction
\begin{equation}\nonumber
  \begin{aligned}
  N_1 & =1 \\
  N_2 & =N^* \rightarrow=0.1267 \\
  I_0 & =\frac{(0.2)(1000)^2}{(1)(7.590)\left(90 \times 10^6\right)(0.04)^3} \rightarrow=4.575 \\
  N & =\frac{0.2}{(0.1267)(2500)(0.04)^3} \rightarrow=9.866 \\
  h_{\text {pen }} & =(0.04)\left(\frac{2}{\pi}(9.866) \ln \left(1+\frac{(4.575)}{(9.866)}\right)\right) \\
  & =0.09571 \text { or } 9.571 \mathrm{~cm}
  \end{aligned}
  \end{equation}
\cite{chen2002deep} states that the formula is valid if the penetration depth is larger than the projectile diameter and the projectile nose length while projectile remains rigid without noticeable deformation and damage.

\subsection*{Steel panel}
The formula proposed by \cite{chen2002deep} has a good agreement with tests on metal, concrete and soil samples with variable nose shapes and velocities. From \cite{li2003dimensionless},
\begin{equation}\nonumber
\begin{aligned}
  \frac{X}{d}=&\frac{2}{\pi} N \ln \left(1+\frac{I}{N}\right)\\
  I=&\frac{\lambda \Phi_{\mathrm{J}}}{A N_1} ; \quad N=\frac{\lambda}{B N_2} ; \quad \lambda=\frac{M}{\rho_0 d^3} ; \quad \Phi_{\mathrm{J}}=\frac{\rho_0 V_{\mathrm{i}}^2}{\sigma_{\mathrm{y}}}
\end{aligned}
  \end{equation}
Based on \cite{forrestal1988dynamic}, \cite{li2003dimensionless} suggests the following for an elastic, perfectly plastic material, where $\gamma$\footnote{About 0.30 for most steel grades.} is the Poisson’s ratio:
\begin{equation}\nonumber
A=\frac{2}{3}\left\{1+\ln \left[\frac{{E}}{3(1-{\gamma}) {\sigma_y}}\right]\right\}
\end{equation}
And for incompressible materials $B=1.5$. Note that parameters $\phi_0, N^*, N_1, N_2$ just depend on the penetrator and not the target material.
So considering sliding friction,
\begin{equation}\nonumber
  \phi_0  =0.9273 \text { or } 53.130^{\circ} ; \quad N^*=0.1267 ; \quad N_1=1.8175 ; \quad N_2=0.1871;
\end{equation}
and
\begin{equation}\nonumber
  \begin{aligned}
  A  =&\frac{2}{3}\left\{1+\ln \left[\frac{{200 \times 10^9}}{3(1-{0.30}) {400 \times 10^6}}\right]\right\} \rightarrow=4.315 \\
  \lambda  =&\frac{0.2}{7850(0.04)^3} \rightarrow=0.3981 \\
  \Phi_{\mathrm{J}}  =&\frac{(7850)(1000)^2}{{400 \times 10^6}} \rightarrow=19.62 \\
  I  =&\frac{(0.3981)(19.62)}{(4.315)(1.817)} \rightarrow=0.9962 \\
  N =& \frac{(0.3981)}{(1.5)(0.1871)} \rightarrow=1.418 \\
  X  =&(0.04)\left(\frac{2}{\pi}(1.418) \ln \left(1+\frac{(0.9962)}{(1.418)}\right)\right) \rightarrow=0.01922 \text { or } 1.922 \mathrm{~cm}
  \end{aligned}
\end{equation}
While without friction
$$  \phi_0=0.9273 \text { or } 53.13^{\circ} ; \quad N^*=N_2 \rightarrow=0.1267 ; \quad N_1=1 ; \quad A=4.315 ; $$
$$  \lambda=0.3981 ; \quad \Phi_{\mathrm{J}}=19.62$$
then
\begin{equation}\nonumber
\begin{aligned}
I=&\frac{(0.3981)(19.625)}{(4.315)(1)}\rightarrow=1.811\\
N=&\frac{(0.3981)}{(1.5)(0.1267)}\rightarrow=2.095\\
X=&(0.04)\left(\frac{2}{\pi}(2.095)\ln{\left(1+\frac{1.811}{2.095}\right)}\right)\rightarrow=0.03323 ~ \mathrm{or}~ 3.323\,\mathrm{cm}
\end{aligned}
\end{equation}
\subsection*{Final design}
Results from earlier discussions are presented in \cref{tab03}.

\begin{table}
    \caption{Summary of results and costs based on \cref{tab02}.}
\label{tab03}
  \resizebox{\textwidth}{!}{%
  \begin{tabular}{ccccccc}
    \toprule
    Panel type&$X$ with(without) friction& Design&Cost\\
  & cm&cm(in)&\$/m$^2$\\
    \midrule
    Concrete\citep{forrestal1994empirical,forrestal1996penetration,frew2006effect,frew1998penetration}&-(17.11)&30(12)&900.0\\
    \midrule
    Concrete\citep{chen2002deep,li2003dimensionless}&5.440(9.571)&16(7)&480.0\\
    \midrule
    Steel\citep{chen2002deep,li2003dimensionless}&1.922(3.323)&6(2.5)&270.0\\
    \bottomrule
    \end{tabular}%
    }
\end{table}

Suprisingly UHPC proves more expensive. Design thickness suggestions were made accounting for 25\% increase due to formula deviations and including a 1.2 safety factor which rises the margin to about 50\% in order to prevent spalling and scabbing as suggested by \cite{krauthammer2008modern}.

\section*{Problem 2}
A 200 kg cased cylindrical charge is set off 45 m away from the concrete face of a two layered blast wall. Design this wall for the largest primary fragments, and also calculate wall ballistic thickness. The PETN charge is encased in a mild steel casing 40 cm long with $d_i/t_c=10$.
\subsection*{Solution}
The main assumption is that the fragment is stopped by penetrating the steel layer thus peforating the concrete layer. 

\begin{table}\centering\caption{Given.}\label{tab04}
\begin{tabular}{cc}
\toprule
Item&Value\\
\midrule
Charge weight, kg(lbs)&200(441)\\
\midrule
Distance, m(ft)&45(150)\\
\midrule
Casing length, cm(in)&40(16)\\
\bottomrule
\end{tabular}
\end{table}

The specific weight and Gurney energy constant for PETN are respectively 0.0635 lb/in$^3$ and 9,600 ft/sec based on \cite{ufc3340022014} while specific weight of mild steel casing is assumed to be about 490 lb/ft$^3$.

Total weight of cased charge equals the sum total of charge and casing weights. Thus considering a cylindrical shape
\begin{equation}\nonumber\begin{aligned}
W_\mathrm{Total}=&W_{ACT}+W_c\rightarrow \rho_\mathrm{PETN}V_{ACT}+\rho_cV_c\\
=&l\left(\rho_\mathrm{PETN}\left(\frac{\pi}{4}\right)d_i^2+\rho_c\left(\frac{\pi}{4}\right)\left(d_i+t_c^2-d_i^2\right)\right)\\
=&t_c^2l\left(\frac{\pi}{4}\right)\left(100\rho_\mathrm{PETN}+44\rho_c\right)\\
\rightarrow t_c=&0.115(1.38)\, \mathrm{ft(in)}\rightarrow d_i = 1.15(13.8)\, \mathrm{ft(in)}\\
W_{ACT}=&16(0.0635)\left(\frac{\pi}{4}\right)\left(13.8\right)^2, W_c=290\,\mathrm{lbs}\\
W=&1.2W_{ACT}\rightarrow=181\,\mathrm{lbs}
\end{aligned}
\end{equation}
Initial velocity of primary fragments resulting from a higher-order detonation of a cylindrical casing with evenly distributed explosives is expressed as\citep{ufc3340022014}
\begin{equation}\nonumber
v_o= \sqrt{2E^\prime\left(\frac{W/W_c}{1+0.5W/W_c}\right)}=9600\sqrt{\frac{181/290}{1+0.5\left(181/290\right)}}\rightarrow=6621\,\mathrm{ft/s}
\end{equation}
largest fragment weight $w_f$ is calculated by setting $N_f$ equal to 1,
\begin{equation}\nonumber
\begin{aligned}
N_f=&\frac{8W_ce^{-{w_f}^{1/2}/M_A}}{M^2_A}\rightarrow=1\\
w_f=&\left(M_A\ln\left[\frac{8W_c}{N_fM_A^2}\right]\right)^2, M_A=Bt_c^{5/6}d^{1/3}_i\left(1+\frac{t_c}{d_i}\right)\\
\xrightarrow[\text{\cite{ufc3340022014}}]{\text{From Table 2-7}} B=&0.248\\
M_A=&0.248\left(1.38\right)^{5/6}\left(13.8\right)^{1/3}(1+0.1)\rightarrow=0.8558\\
w_f=&\left(0.8558\ln{\left[\frac{8(290)}{(1)(0.8558)^2}\right]}\right)^2\rightarrow=47.5\,\mathrm{oz}
\end{aligned}
\end{equation}
\cite{ufc3340022014} expresses striking velocity as
\begin{equation}\nonumber
V_s=V_oe^{-12k_VR_f}, k_V=0.5(A/w_f)\rho_aC_D;
\end{equation}
Assuming $A/w_f=0.78/w_f^{1/3}$ for a random mild steel fragment (in$^2$/oz), $\rho_a=0.00071$ oz/in$^3$ and $C_D=1.2$ for primary fragments, thus for a cylindrical casing
\begin{equation}\nonumber
\begin{aligned}
d=&\sqrt{\left(\frac{4}{\pi}\right)0.78w_f^{2/3}}\rightarrow=3.61\, \mathrm{in}\rightarrow A=10.24\,\mathrm{in}^2\\
k_V=&\frac{1}{2}\left(\frac{10.24}{47.5}\right)(0.00071)(1.2)\rightarrow=9.184\times10^{-5}\\
V_s=&(6621)e^{-12(9.184\times10^{-5})(150)}\rightarrow=5612\,\mathrm{ft/s}\,\text{(Case I)}
\end{aligned}
\end{equation}
Striking velocity is determinable assuming values like shape factor $N$ and fragment nose tangent ogive caliber radius $n$ for an alternate fragment form. Assuming $N=1$ and $n=1.5$, according to Figure 2-244b in \cite{ufc3340022014} 
\begin{align}\nonumber
V_f=&1.2d^3, \rho_f = 4.6~\mathrm{oz/in}^3\rightarrow w_f= 5.52d^3\,\mathrm{oz} \rightarrow d=2.05\,\mathrm{in}\rightarrow A=3.30\, \mathrm{in}^2;\\\nonumber
k_V=&\frac{1}{2}\left(\frac{3.30}{47.5}\right)(0.00071)(1.2)\rightarrow=2.960\times10^{-5}\\\nonumber
V_s=&(6621)e^{-12(2.960\times10^{-5})(150)}\rightarrow=6277\,\mathrm{ft/s}\,\text{(Case II)}
\end{align}

Perforating the concrete layer the fragment is embeded penetrating in the steel layer
\begin{equation}\nonumber\begin{aligned}
X_f=&\left\{\begin{array}{lr}
4.0\times10^{-3}\sqrt{K\,N\,D}d^{1.1}v_s^{0.9},& X_f\leq2d\\
4.0\times10^{-6}K\,N\,Dd^{1.2}v_s^{1.8}+d,& X_f>2d\\
\end{array}\right.\\
K=&\sqrt{12.91}{\sqrt{f_c'}}\rightarrow \frac{12.91}{\sqrt{13053}}=0.1130\\
\end{aligned}
\end{equation}
Penetration depth is larger than $2d$ for both cases.
\subsubsection*{Case I}
\begin{align}\nonumber
 N=&1, D=47.5/(2.05)^3\rightarrow =5.51\\
X_f=&4.0\times10^{-6}(0.113)(1)(5.51)(2.05)^{1.2}(6277)^{1.8}+2.05\rightarrow=42.45\,\mathrm{in}\nonumber\\
\xrightarrow[\text{for }f'_c]{\text{Correcting}} X'_f=&X_f\sqrt{\frac{4,000}{f_c'}}\rightarrow(42.45)\sqrt{\frac{4000}{13053}}=23.5\,\mathrm{in}\nonumber
\end{align}
An armor piercing fragment woul produce this level of penetration and having a mild steel casing sanctions another correction based off of Table 4-16 from \cite{ufc3340022014}.
\begin{align}\nonumber
X'_f=&kX_f\rightarrow X_\mathrm{pen}\,\text{or}\, X_f=0.7\times(23.5)\rightarrow=16.5\,\mathrm{in}\\\nonumber
T_{pf}=&1.13X_fd^{1/10}+1.311d\rightarrow=1.13(16.5)(2.05)^{1/10}+1.311(2.05)=22.35\,\mathrm{in}\nonumber
\end{align} 
Considering $X_f>2d$ \cite{ufc3340022014} expresses fragment residual velocity after perforating a concrete wall $T_c$ thick as
\begin{equation}\label{eqvvr}
V_r/V_s=\left[1-\frac{T_c}{T_{pf}}\right]^{0.555}, X_f>2d
\end{equation}
In addition \cite{ufc3340022014} provides the follwoing for mild steel plate penetration that in our case the striking velocity would equal the concrete wall fragment residual velocity. Note that resulting thickness values are ballistic limits and design should include certain factors of safety.
\begin{figure}\centering\caption{Wall layout.}\label{fig01}
\begin{tikzpicture}
\draw (1,2) rectangle (4,3) node [pos=0.5]{Fragment, $V\rightarrow$};
\draw (5,0) rectangle (10,5) node [pos=0.5,rotate=90] {Concrete};
\draw (10,5) rectangle (11,0) node [pos=0.5,rotate=90]{Steel Plate};
\end{tikzpicture}
\end{figure}
\begin{equation}\label{eqxs}
x=0.21w_f^{0.33}V_s^{1.22}
\end{equation}
Given that $T_C/T_s=5$ and $v_{r\,\text{(Concrete)}}=v_{s\text{(Steel)}}$ and that the striking velocity in the above formula is expressed in kft/sec after substituting \cref{eqvvr} into \cref{eqxs} with known parameters the resulting equation is solved for $T_s$,
\begin{align}\nonumber
V_r=&6277\left[1-\frac{5T_s}{22.35}\right]^{0.555}\rightarrow x=0.21w_f^{0.33}V_s^{1.22}\\\nonumber
 x=&0.21(47.5)^{0.33}\left(6.277\left[1-\frac{5T_s}{22.35}\right]^{0.555}\right)^{1.22}\rightarrow=3.13\,\mathrm{in}\\\nonumber
 \text{ and thus }\rightarrow T_c=&15.65\,\mathrm{in}
\end{align}
The wall would end up $T_T=T_c+T_s=18.8~\mathrm{in}$ thick.
\subsubsection*{Case II}
Following the same previous steps,
\begin{align}\nonumber
N=&1, D=47.5/(3.61)^3\rightarrow=1.01\\\nonumber
X_f=&4.0\times 10^{-6}(0.113)(1)(1.01)(3.61)^{1.2}(5612)^{1.8}+3.61\rightarrow=15.55\,\mathrm{in}\\\nonumber
X'_f=&X_f\sqrt{\frac{4000}{f'_c}}\rightarrow 15.55\sqrt{\frac{4000}{13053}}=8.61\,\mathrm{in}\\\nonumber
X_\mathrm{Pen}\text{ or } X_f=&0.7(8.61)=6.03\,\mathrm{in}\\\nonumber
T_{pf}=&1.13(6.03)(3.61)^{0.1}+1.311(3.61)\rightarrow=12.48\,\mathrm{in}\\\nonumber
v_r=&5612\left[1-\left(\frac{5T_s}{12.48}\right)\right]^{0.555}\, \text{ for } X_f<2d\\\nonumber
T_s=&0.21(47.5)^{0.33}\left[5.612\left[1-\frac{5T_s}{12.48}\right]^{0.555}\right]^{1.22}\rightarrow=2.02\,\mathrm{in}\rightarrow T_c=10.1\,\mathrm{in}
\end{align}
And in this case the wall will end up at least $T_T=12.12\,\mathrm{in}$ thick.

      \bibliographystyle{plainnat}
      \bibliography{bibs}
\end{document}
